\section{Coscienza concretezza che non ho visto}

\subsection{Essere me ''leggere''}

\begin{itemize}
    \item \cite{limeseurogeuroitalia}. \keyword{Leggo come} non ne avessi bisogno: pausa caf\'e a met\'a. Globalizzazione: vapore/trasporti, informazione/segmentazione produzione. Germania: assembla per esportare nel mondo. Declino italiano: crisi lira '92 e declino produttivit\'a (disoccupazione): divario nord-sud; crisi 2008 recessione 2010. Unione monteria: germania contraria a controllo europea sulle banche dei lander. Conseguenze di uscita dell'italia sui paesi del kerneuro.
    
\end{itemize}

\section{tentativo fusione dovere/piacere}

\begin{itemize}
\item Esperimental foundation particle physics (''photon neutron electron'', )
\item ''La naus\'ee'' (Fuillet sans date)
\item gomorra (narrazione-il porto)
\item teeteto (introduzione di salvatore natoli)

\end{itemize} 

\section{''carmilla''}

\subsection{Nuovi volti dei movimenti sociali: dalle lotte sul territorio ai ''cortei di testa''.}

In alcune grandi città francesi, durante le manifestazioni di strada della primavera 2016 contro la ''Loi Travail'', cosa spingeva tanta gente d’ogni et\'a e categoria sociale a risalire i marciapiedi o a uscire dai ranghi inquadrati dalle organizzazioni sindacali per unirsi a quello che si \'e rapidamente auto-battezzato ''corteo di testa''? Cosa li incitava a raggiungere questa componente che, da una scadenza all’altra, \'e cresciuta fino a comprendere varie migliaia di persone e a costituire, talora, la met\'a della manifestazione?

Era peraltro l\'a che si subivano i bombardamenti intensivi di lacrimogeni, le incursioni delle Brigate anti-crimine per fermare individui secondo criteri sconosciuti, i getti delle motopompe. Era l\'a che si \keyword{rischiava} di perdere un occhio per effetto di un lancio di flashball o di cadere in coma a causa di una granata detta ''di disimpegno''. Dev’essere accaduto qualcosa perch\'e tanta gente si sia esposta volontariamente a simili pericoli.

Se si deve cercare una novità decisiva del movimento della primavera 2016, non sta forse in quelle ''Notti in piedi'' su cui si \'e concentrata l’attenzione dei media, in Francia e soprattutto all’estero. I raduni notturni in certe piazze, a imitazione del 15M spagnolo o di Occupy Wall Street, hanno certo rivelato un bisogno di comunicazione diretta e liberato spazi di espressione collettiva. La loro estensione a tutto il paese ha mostrato l’ampiezza di questo bisogno e la necessit\'a di questa espressione. Ma, a differenza degli omologhi statunitensi e spagnoli, il ''movimento delle piazze'' \'e rimasto confinato in limiti temporali quotidiani, che si sono presto rivelati limiti e basta. Promosso sulla bella idea di un mese di marzo destinato a durare oltre il 31, la sua pretesa di cambiare la costituzione o la societ\'a discutendo dalle 21 alle 6, su autorizzazione rinnovabile della prefettura, ha giocoforza avuto fin dall’inizio qualcosa di farsesco. Padrone del tempo \'e rimasto lo Stato, che durante il giorno riprendeva le piazze. Bench\'e la composizione sociale delle Notti in piedi sia stata pi\'u varia di quanto abbiano preteso diverse destre, ansiose di classificarle nella categoria onnicomprensiva dei ''radical chic'', i loro orari notturni e, malgrado alcuni tentativi in direzione delle periferie, il loro confinamento nel centro cittadino, hanno limitato la partecipazione dei ceti popolari. Se sono state talora sede di dibattiti opportuni, le assemblee generali troppo spesso sono somigliate a gruppi di ascolto, in cui la sfilata delle soggettività sofferenti rimane senza conseguenze. In definitiva, sono state soprattutto utili come luoghi di preparazione di manifestazioni selvagge o di interventi sulle lotte in corso.

Ci\'o che molti cercavano, partendo dalle piazze occupate per andare a sostenere i ferrovieri in sciopero o i migranti che occupavano un liceo abbandonato – vale a dire una conflittualit\'a non limitata alle parole e comune a soggetti sociali variegati – la trovavano nei ''cortei di testa''. Oltre la presenza, tradizionale, di giovani in cerca di scontro (con la novità di un calo sensibile dell’et\'a media), si notava una forte rappresentanza di pensionati attivi, nonch\'e di persone di et\'a matura che esibivano i segni esteriori (spille, caschi, bandiere) dell’appartenenza alla classe operaia sindacalizzata. Questa componente maggioritaria del ''corteo di testa'' si distingueva per la sua attitudine, se non attivamente complice, quanto meno per niente ostile verso coloro che i media dominanti chiamavano ''casseurs''.

Quei manifestanti sembravano quesi tutti insensibili all’argomento ordinario secondo il quale rompere le vetrine delle banche squalificherebbe il movimento presso l’opinione pubblica. Tutto accadeva come se avessero rinunciato a preoccuparsi della reazione dei media che, in assenza di ''casseurs'', direbbero in ogni caso che la manifestazione era meno importante della precedente e che il movimento si stava sgonfiando. Nessuno si indignava per i muri coperti di slogan, n\'e sembrava considerare le agenzie bancarie e immobiliari dei bersagli illegittimi.

Il fatto che la maggioranza non si unisca all’azione mostra chiaramente che essa \'e trattenuta non tanto dalla paura, quanto dal fatto che, per correre maggiori rischi, avrebbe bisogno di far di meglio che rompere vetri: forse nuocere realmente alle banche e all’ordine economico. Questo non impediva i gesti di solidarietà verso chi, mascherato, subiva la repressione, e non sono mancati episodi in cui sindacalisti e sconosciuti di una certa et\'a hanno tentato di liberare i presunti ''casseurs'' dalle mani della polizia.

Ci\'o che colpiva, nel corteo di testa, non era tanto la somiglianza con i movimenti sociali anteriori (dallo sciopero contro la ''riforma'' della sicurezza sociale del 1995 a quello contro la ''riforma'' del sistema pensionistico del 2010), quanto la similitudine con conflitti nati lontano dalle grandi città e ancorati a territori precisi. Coraggio e determinazione di fronte alla repressione, creatività dei modi espressivi e d’intervento, eterogeneità delle pratiche e degli attori sono anche i tratti salienti dell’opposizione alla costruzione dell’aeroporto di Notre-Dame-des-Landes, cristallizzata attorno alla Zona da Difendere (ZAD), e del movimento NO-TAV.

Quando il corteo di testa si avviava, di fronte ai ranghi serrati delle forze dell’ordine e dei loro furgoni, non avrebbe stupito, davanti alla risoluzione gioiosa delle sue file, sentire urlare: A sar\'a d\:ura! E’ il grido di adunata, in lingua piemontese, delle popolazioni della Valle di Susa in lotta contro il progetto di nuovo Treno ad Alta Velocità Lione-Torino. Annuncia che lo scontro sar\'a duro – sottinteso ''per noi'', ma anche ''per loro''.

Il 17 novembre 2016, al termine di un maxiprocesso d’appello contro 53 attivisti NO TAV, il tribunale di Torino ne ha condannati 38 a pene di prigione varianti da alcuni mesi a quattro anni e mezzo. La loro incriminazione era legata a due episodi della saga di questa lotta che dura da 25 anni: l’evacuazione, nel giugno 2011, della ''libera repubblica della Maddalen'', accampamento installato in una zona di estensione del cantiere, e l’imponente manifestazione che ne era seguita, in luglio, col tentativo di riprendersi il terreno.

L’accusa non ha esitato a sostenere la colpevolezza di una persona che sarebbe stata vista in due luoghi diversi nello stesso tempo, n\'e a ricorrere all’imputazione, molto vaga ma pratica, di ''concorso morale''. Un fatto fra i tanti dell’interminabile serie di denunce e di limitazioni alla libert\'a di circolare. Una delle personalit\'a storiche del movimento, Nicoletta Dosio, \'e divenuta la figura di punta della resistenza agli innumerevoli arresti domiciliari: avendo rifiutato di sottomettersi alle restrizioni, si \'e ritrovata davanti a un tribunale in cui il procuratore ha chiesto per lei otto mesi di carcere.