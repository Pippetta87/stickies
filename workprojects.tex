\chapter{Programmin}

\section{latex(compiled) plots}

\begin{itemize}

\item 3D plot of addensamento nello spazio degli elementi propri degli asteroidi

\item Plot curve with pgfplots intersecting in one point
\begin{verbatim}
\begin{tikzpicture}
\begin{axis}
[
title={p-value},
xmin=0,xmax=3,ymin=0,ymax=3,
]
\draw (0,1) -- (1,1);
\draw[dashed] (0,2) .. controls (0.2,0.5) .. (1,0);
\end{axis}
\end{tikzpicture}
\end{verbatim}

\item Banda di confidenza: superficie 3d tagliata da piani, banda sul piano xy, plot 3d con dominio definito da disuguaglianza - google fu: pgfplot 3d plot contours. (\cite{tikzgnuplotparam}: parameters in addplot wit raw gnuplot; \cite{tikzgnuplotscript}: use gnuplot script in addplot)
gnuplot script graphing 3d slices:

\begin{verbatim}
# set terminal png transparent nocrop enhanced font arial 8 size 420,320 
# set output 'surface1.17.png'
set dummy u,v
set label 1 "increasing v" at 6, 0, -1 left norotate back nopoint
set label 2 "increasing u" at 0, -5, -1.5 left norotate back nopoint
set label 3 "floor(u)%3=0" at 5, 6.5, -1 left norotate back nopoint
set label 4 "floor(u)%3=1" at 5, 6.5, 0.100248 left norotate back nopoint
set arrow 1 from 5, -5, -1.2 to 5, 5, -1.2  back nofilled linetype -1 linewidth 1.000
set arrow 2 from -5, -5, -1.2 to 5, -5, -1.2  back nofilled linetype -1 linewidth 1.000
set arrow 3 from 5, 6, -1 to 5, 5, -1  back nofilled linetype -1 linewidth 1.000
set arrow 4 from 5, 6, 0.100248 to 5, 5, 0.100248  back nofilled linetype -1 linewidth 1.000
set parametric
set view 70, 20, 1, 1
set samples 51, 51
set isosamples 30, 33
set hidden3d offset 1 trianglepattern 3 undefined 1 altdiagonal bentover
set ztics border nomirror norotate -1.00000,0.25,1.00000
set title "\"fence plot\" using single parametric surface with undefined points" 0.000000,0.000000  font ""
set xlabel "X axis" -3.000000,-2.000000  font ""
set xrange [ -5.00000 : 5.00000 ] noreverse nowriteback
set ylabel "Y axis" 3.000000,-2.000000  font ""
set yrange [ -5.00000 : 5.00000 ] noreverse nowriteback
set zlabel "Z axis" -5.000000,0.000000  font ""
set zrange [ -1.00000 : 1.00000 ] noreverse nowriteback
sinc(u,v) = sin(sqrt(u**2+v**2)) / sqrt(u**2+v**2)
xx = 6.08888888888889
dx = 1.11
x0 = -5
x1 = -3.89111111111111
x2 = -2.78222222222222
x3 = -1.67333333333333
x4 = -0.564444444444444
x5 = 0.544444444444445
x6 = 1.65333333333333
x7 = 2.76222222222222
x8 = 3.87111111111111
x9 = 4.98
xmin = -4.99
xmax = 5
n = 10
zbase = -1
splot [u=.5:3*n-.5][v=-4.99:4.99] 	 xmin+floor(u/3)*dx, v, ((floor(u)%3)==0) ? zbase : 			 (((floor(u)%3)==1) ? sinc(xmin+u/3.*dx,v) : 1/0) notitle
\end{verbatim}

\begin{verbatim}
# set terminal png transparent nocrop enhanced font arial 8 size 420,320 
# set output 'surface1.16.png'
set dummy u,v
set label 1 "increasing v" at 6, 0, -1 left norotate back nopoint
set label 2 "u=0" at 5, 6.5, -1 left norotate back nopoint
set label 3 "u=1" at 5, 6.5, 0.100248 left norotate back nopoint
set arrow 1 from 5, -5, -1.2 to 5, 5, -1.2  back nofilled linetype -1 linewidth 1.000
set arrow 2 from 5, 6, -1 to 5, 5, -1  back nofilled linetype -1 linewidth 1.000
set arrow 3 from 5, 6, 0.100248 to 5, 5, 0.100248  back nofilled linetype -1 linewidth 1.000
set parametric
set view 70, 20, 1, 1
set samples 51, 51
set isosamples 2, 33
set hidden3d offset 1 trianglepattern 3 undefined 1 altdiagonal bentover
set ztics border nomirror norotate -1.00000,0.25,1.00000
set title "\"fence plot\" using separate parametric surfaces" 0.000000,0.000000  font ""
set xlabel "X axis" -3.000000,-2.000000  font ""
set xrange [ -5.00000 : 5.00000 ] noreverse nowriteback
set ylabel "Y axis" 3.000000,-2.000000  font ""
set yrange [ -5.00000 : 5.00000 ] noreverse nowriteback
set zlabel "Z axis" -5.000000,0.000000  font ""
set zrange [ -1.00000 : 1.00000 ] noreverse nowriteback
sinc(u,v) = sin(sqrt(u**2+v**2)) / sqrt(u**2+v**2)
xx = 6.08888888888889
dx = 1.10888888888889
x0 = -5
x1 = -3.89111111111111
x2 = -2.78222222222222
x3 = -1.67333333333333
x4 = -0.564444444444444
x5 = 0.544444444444445
x6 = 1.65333333333333
x7 = 2.76222222222222
x8 = 3.87111111111111
x9 = 4.98
splot [u=0:1][v=-4.99:4.99] 	x0, v, (u<0.5) ? -1 : sinc(x0,v) notitle, 	x1, v, (u<0.5) ? -1 : sinc(x1,v) notitle, 	x2, v, (u<0.5) ? -1 : sinc(x2,v) notitle, 	x3, v, (u<0.5) ? -1 : sinc(x3,v) notitle, 	x4, v, (u<0.5) ? -1 : sinc(x4,v) notitle, 	x5, v, (u<0.5) ? -1 : sinc(x5,v) notitle, 	x6, v, (u<0.5) ? -1 : sinc(x6,v) notitle, 	x7, v, (u<0.5) ? -1 : sinc(x7,v) notitle, 	x8, v, (u<0.5) ? -1 : sinc(x8,v) notitle, 	x9, v, (u<0.5) ? -1 : sinc(x9,v) notitle
\end{verbatim}

\begin{verbatim}
\begin{frame}{Costruzione Neyman per pdf uniforme}
\begin{tikzpicture}[scale=1.5]
\begin{axis}[
%view       = {-5}{-5},
axis lines = middle,
xlabel=$x_M$, ylabel=$m$, zlabel=$\prob{(x_M;m)}$
]
\addplot3 [surf,
opacity=0.1
] gnuplot[id=xMband,raw gnuplot,surf] {N=20; f(x,y)=(x>y) ? 1/0:(N/y)*(x/y)^(N-1); splot [1:10][1:10] ((N/y)*(x/y)^(N-1))};
\end{axis}
\end{tikzpicture}
\end{frame}
\end{verbatim}
\item banda di confidenza per parametro pdf uniforme co ordinamento LR:
\begin{verbatim}
\begin{tikzpicture}[scale=0.3]%[myscalar]
\pgfmathsetmacro{\mparam}{10}
\pgfmathparse{\mparam*(0.9+*0.1*rand)}
\pgfmathsetmacro{\xmrv}{\pgfmathresult}
\begin{axis}[title={Banda confidenza $U([0,m])$: ordinamento $\lr{}$},name=LRuniformband,ymin=0,xmin=0,xmax=10,xlabel={$x_M$},ylabel={$\lr{}$},y label style={rotate=-90, at={(-0.1,1.)}},extra x tick style={% changes for extra x ticks
	tick label style={yshift=0mm}},%extra x ticks={\xmrv},extra x tick labels={measure $x_0$}
]
\addplot[name path global=lr, no marks,raw gnuplot] gnuplot[id=lr,domain=0:10] {N=100;m=10; plot (x/\mparam)^N};
%set xtics add ("x0" (0.9*m+m*0.1*rand(0)));
%\addplot[name path global=bm, no marks] gnuplot[id=lowerbound,domain=0:8] {x};
%\addplot[name path global=meas,dash pattern=on 4pt off 3pt,no marks,black] gnuplot[domain=0:8] {0.3};
%\path[name intersections={of={bp and meas},name=i},name intersections={of={bm and meas},name=j}] (i-1) (j-1);
%\pgfplotsextra{\path (i-1) \pgfextra{\markxof{i-1}\xdef\myfirsttick{\pgfmathresult}}(j-1); \pgfextra{\markxof{j-1}\xdef\mysecondtick{\pgfmathresult}};}
\end{axis}
%\draw[ultra thin, draw=gray] (i-1 |- {rel axis cs:0,0}) node[fill=yellow,yshift=-1ex]{\pgfmathprintnumber[fixed,precision=5]\myfirsttick} -- (i-1);
%\draw[ultra thin, draw=gray] (j-1 |- {rel axis cs:0,0}) node[fill=red,below]{\pgfmathprintnumber[fixed,precision=5]\mysecondtick} -- (j-1);
\end{tikzpicture}
\end{verbatim}
\end{itemize}

\section{systemd: battery monitor}

\section{odbII ios app}

\chapter{unipi}

\chapter{Ruffolo}

\chapter{In tenda/macchina}